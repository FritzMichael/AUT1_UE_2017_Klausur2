\documentclass[crop=false]{standalone}
\usepackage[utf8]{inputenc}
\usepackage{amsmath}
\usepackage[dvipsnames]{xcolor}
\usepackage{pdfpages}
\usepackage{enumerate}
\usepackage{amssymb}
\usepackage[framemethod=default]{mdframed}
\usepackage[nomarginpar,left=2cm,right=2cm,top = 2cm, bottom = 2cm]{geometry}

\renewcommand{\thesubsection}{\thesection.\alph{subsection}}
\renewcommand{\thesubsubsection}{\thesection.\alph{subsection}.\roman{subsubsection}}

\mdfdefinestyle{theoremstyle}{%
linecolor=black,linewidth=.3pt,%
frametitlerule=true,%
frametitlebackgroundcolor=blue!5,
innertopmargin=\topskip,nobreak=true,
}

\mdfdefinestyle{style2}{frametitle={},%
             linewidth=.3pt,topline=true,backgroundcolor=blue!3!green!8!}

\mdtheorem[style=theoremstyle]{task}{Angabe}

\newmdenv[style = style2,title=false]{solution}

\begin{document}
\begin{task}[Beobachtbarkeit und Beobachterentwurf]
Für das LTI-System
\[\begin{aligned} \dot{\mathbf{x}} &=\left[\begin{array}{cc}{2} & {0} \\ {-2} & {-1}\end{array}\right] \mathbf{x}+\left[\begin{array}{c}{1} \\ {0}\end{array}\right] u \\ y &=\left[\begin{array}{cc}{2} & {0}\end{array}\right] \mathbf{x}-3 u \end{aligned}\]
soll ein vollständiger Beobachter entworfen werden. Bearbeiten Sie dazu folgende Unterpunkte
\begin{enumerate}[i]
    \item Können die Eigenwerte des Fehlersystems durch geeignete Wahl von $\hat{\mathrm{k}}$ beliebig
platziert werden? Falls NEIN, geben Sie die nicht beeinflussbaren Eigenwerte des
Fehlersystems an. \emph{Begründung!}
\begin{solution}
Die Eigenwerte des Fehlersystems können genau dann durch geeignete Wahl von $\hat{\mathrm{k}}$ beliebig platziert werden, wenn das System vollständig beobachtbar ist.

Prüfung der Beobachtbarkeit:
\[\mathbf{M}_O = 
     \begin{pmatrix} \mathbf{c}^T \\ \mathbf{c}^T \mathbf{A} \end{pmatrix} = 
     \begin{pmatrix} 2 & 0 \\ 4 & 0  \end{pmatrix}, \quad
     \text{Rank}\left(\mathbf{M}_O\right) = 1 \neq n \rightarrow \text{System ist nicht vollst. beobachtbar}
\]
Das System ist bereits in einen beobachtbaren und einen nicht-beobachtbaren Teil zerlegt. 
Der Eigenwert des nicht-beobachtbaren Teilsystems, der nicht verändert werden kann, liegt bei $-1$ (Untermatrix $\mathbf{A}_{22}$).
\end{solution}
\item Geben Sie das Fehlersystem für die Wahl $\hat{\mathbf{k}}=\left[\begin{array}{cc}{-2} & {2}\end{array}\right]^{\top}$ an.
\begin{solution}
Das Fehlersystem ist definiert als:
\[ 
\dot{\mathbf{e}}=\left(\mathbf{A}+\hat{\mathbf{k}} \mathbf{c}^{T}\right) \mathbf{e},
\quad \mathbf{A}+\hat{\mathbf{k}} \mathbf{c}^{T} = 
\begin{pmatrix}
2 & 0 \\
-2 & -1
\end{pmatrix} +
\begin{pmatrix}
-4 & 0 \\
4 & 0
\end{pmatrix}
=
\begin{pmatrix}
-2 & 0 \\
2 & -1
\end{pmatrix}
 \]
\end{solution}
\item Der Beobachter wird als dynamisches System der Form
\[\begin{aligned} \dot{\mathbf{z}} &=\mathbf{A}_{B} \mathbf{z}+\mathbf{B}_{B} \mathbf{w} \\ \hat{\mathbf{x}} &=\mathbf{C}_{B} \mathbf{z}+\mathbf{D}_{B} \mathbf{w} \end{aligned}\]
mit dem Zustand z und dem Eingang w implementiert. Geben Sie w sowie die
Matrizen $\mathbf{A}_{B}, \mathbf{B}_{B}, \mathbf{C}_{B}$ und $\mathbf{D}_{B}$ unter Verwendung von $\hat{\mathbf{k}}$ aus Punkt (ii) explizit
an.

\begin{solution}
Der Eingang $\mathbf{w}$ des Beobachters besteht aus dem Eingang des Systems und dem Ausgang des Systems:
\[\mathbf{w} = \begin{pmatrix}
u \\ y
\end{pmatrix}
\]
Der Zustandsvektor $\mathbf{z}$ des Beobachters entspricht einer Schätzung des Zustandsvektors $\mathbf{x}$:
\[\mathbf{z} = \begin{pmatrix}
z_1 \\ z_2
\end{pmatrix}
\]
Auflistung der Matrizen:
\[
\begin{aligned}
\mathbf{A}_B &= \mathbf{A}+\hat{\mathbf{k}} \mathbf{c}^{T} = \begin{pmatrix}
-2 & 0 \\
2 & -1
\end{pmatrix}, &\quad
\mathbf{B}_B &= 
\begin{pmatrix}
\mathbf{b}+\hat{\mathbf{k}} d &  - \hat{\mathbf{k}}
\end{pmatrix}
= \begin{pmatrix}
7 & 2 \\
-6 & -2
\end{pmatrix},\\
\mathbf{C}_B &= \mathbf{E},
&
\mathbf{D}_B &= \mathbf{0}
\end{aligned}
\]
\end{solution}

\end{enumerate}
\end{task}
\end{document}