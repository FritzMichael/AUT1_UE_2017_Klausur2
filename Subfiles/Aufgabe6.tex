\documentclass[crop=false]{standalone}
\usepackage[utf8]{inputenc}
\usepackage{amsmath}
\usepackage[dvipsnames]{xcolor}
\usepackage{pdfpages}
\usepackage{enumerate}
\usepackage{amssymb}
\usepackage[framemethod=default]{mdframed}
\usepackage[nomarginpar,left=2cm,right=2cm,top = 2cm, bottom = 2cm]{geometry}

\renewcommand{\thesubsection}{\thesection.\alph{subsection}}
\renewcommand{\thesubsubsection}{\thesection.\alph{subsection}.\roman{subsubsection}}

\mdfdefinestyle{theoremstyle}{%
linecolor=black,linewidth=.3pt,%
frametitlerule=true,%
frametitlebackgroundcolor=blue!5,
innertopmargin=\topskip,nobreak=true,
}

\mdfdefinestyle{style2}{frametitle={},%
             linewidth=.3pt,topline=true,backgroundcolor=blue!3!green!8!}

\mdtheorem[style=theoremstyle]{task}{Angabe}

\newmdenv[style = style2,title=false]{solution}

\begin{document}
\begin{task}
Betrachten Sie das LTI-System
\[ 
\begin{aligned} \dot{\mathbf{x}} &=\mathbf{A} \mathbf{x}+\mathbf{b} u \\ y &=\mathbf{c}^{\top} \mathbf{x} \end{aligned}
 \]
mit den Matrizen
\[ 
\mathbf{A}=\left[\begin{array}{ccc}{0} & {1} & {0} \\ {0} & {-1} & {1} \\ {-1} & {-1} & {0}\end{array}\right], \quad \mathbf{b}=\left[\begin{array}{c}{1} \\ {2} \\ {0}\end{array}\right], \quad \mathbf{c}^{T}=\left[\begin{array}{lll}{1} & {1} & {0}\end{array}\right]
 \]
 \begin{enumerate}[i]
     \item Bestimmen Sie den nichtbeobachtbaren Unterraum $\mathcal{O}^{\perp} \subseteq \mathbb{R}^{3}$
     \begin{solution}
     \[\mathcal{O}^{\perp} =  \text{Ker}\left(\mathbf{M}_O\right); \ \mathbf{M}_O = 
     \begin{pmatrix} \mathbf{c}^T \\ \mathbf{c}^T \mathbf{A} \\ \mathbf{c}^T \mathbf{A}^2 \end{pmatrix} = 
     \begin{pmatrix} 1 & 1 & 0 \\ 0 & 0 & 1 \\ -1 & -1 & 0 \end{pmatrix}\]
     \[ \mathcal{O}^{\perp}=\text{Ker}\left(\mathbf{M}_O\right) = \text{span}\left\{\begin{pmatrix} 1\\-1\\ 0\end{pmatrix}\right\}\]
     \end{solution}
     \item Prüfen Sie, welche der Transformationen $\mathbf{z}=\mathbf{M}_{i}\mathrm{x}$, $i=1,2,3$ geeignet sind, um
das System in ein nichtbeobachtbares Teilsystem und ein Restsystem zu transformieren.
Begrunden Sie Ihr Auswahl ohne die Transformation explizit durchzuführen!
\[ 
\mathbf{M}_{1}=\left[\begin{array}{ccc}{3} & {3} & {0} \\ {3} & {3} & {1} \\ {0} & {-1} & {0}\end{array}\right], \quad \mathbf{M}_{2}=\left[\begin{array}{ccc}{1} & {2} & {0} \\ {2} & {-2} & {0} \\ {0} & {-3} & {1}\end{array}\right], \quad \mathbf{M}_{3}=\left[\begin{array}{ccc}{1} & {1} & {0} \\ {0} & {0} & {1} \\ {-1} & {-1} & {0}\end{array}\right]
 \]
 \begin{solution}
 Zuerst muss eine Basis für $\mathcal{O}$ gefunden werden:
 \[ \mathcal{O} = \text{span}\left\{ 
 \begin{pmatrix} 1 & 1 & 0 \end{pmatrix},
 \begin{pmatrix} 0 & 0 & 1 \end{pmatrix}
 \right\} \]
 und ein Vektor der diese Basis komplementiert: $\mathbf{v}_3 = \begin{pmatrix} 0 & 1 & 0 \end{pmatrix}$
 
 
 Damit kann die Transformationsmatrix gebildet werden:
 \[ 
\left[\begin{array}{l}{\mathbf{z}_{1}} \\ {\mathbf{z}_{2}}\end{array}\right]=\left[\begin{array}{c}{\mathbf{V}_{1}} \\ {\mathbf{V}_{2}}\end{array}\right] \mathbf{x}, \quad 
\underbrace{
\mathbf{V}_{1} = 
\begin{pmatrix}
1&1&0\\
0&0&1\\
\end{pmatrix}}_{\text{Basisvektoren für }\mathcal{O}}, \quad
\underbrace{
\mathbf{V}_{2} = 
\begin{pmatrix}
0&1&0\\
\end{pmatrix}}_{\text{Komplementärvektoren}}
 \]
 Von den drei Matrizen erfüllt nur die Matrix $\mathbf{M}_1$ diese Struktur.
 
 \underline{Explizite Transformation als Beweis ($\mathbf{M}_1 = \mathbf{V}$):}
 \[ \begin{aligned}\dot{\mathbf{z}} &= \mathbf{V}\mathbf{A}\mathbf{V}^{-1}\mathbf{z}+\mathbf{V}\mathbf{b}u\\
 y &= \mathbf{c}^T \mathbf{V}^{-1}\mathbf{z}
 \end{aligned} \rightarrow
 \begin{aligned}\dot{\mathbf{z}} &= 
 \begin{pmatrix}
 -3 & 3 & 0\\
 \frac{-10}{3} & 3 & 0\\
 1 & -1 & -1
 \end{pmatrix}
 \mathbf{z}+
 \begin{pmatrix}
 9\\9\\-2
 \end{pmatrix}
 u\\
 y &= \begin{pmatrix}
 \frac{1}{3}&0&0
 \end{pmatrix}\mathbf{z}
 \end{aligned}
 \]
 Man erkennt, dass sich der dritte Eintrag des Zustandsvektor nicht auf die ersten Beiden auswirkt (Dynamikmatrix $\mathbf{A}$) und der dritte Zustand nicht direkt messbar ist (Ausgangsmatrix $\mathbf{c}^T$). Somit ist der dritte Zustand nicht beobachtbar.
 \end{solution}
 \end{enumerate}
\end{task}
\end{document}